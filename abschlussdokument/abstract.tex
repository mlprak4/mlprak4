%!TEX root = bare_conf.tex

\begin{abstract}
	In this work, we achieved autonomous driving in a 3D simulation based on deep Q learning. Unlike some other works, which mostly got the self-driving capability in car running games, we utilized a more realistic car model, which can be driven more flexibly by specifying  velocities and steering angles. Instead of capturing the screen we used the raw image from a simulated normal camera in the driver view as input. We evaluated a series of reward functions and found out that the selection of reward function is the most important factor to achieve a successful autonomous driving. Besides distance to middle of lane, the angle between direction of the car and direction of the lane is also essential to make the vehicle run smoothly at curves. Furthermore, velocity of the action set is also a critical component to ge a better performance of training and testing. 
\end{abstract}
